The design and analysis of an IDHP-based flight controller for a Bo-105 helicopter are presented. The controller was shown to be able to reliably learn to directly control the pitch angle and altitude without an offline learning phase. Results from \cite{Heyer2020}, indicating that the addition of a target critic is a valuable addition to the IDHP framework, were shown not to necessarily apply in all situations. Though a target critic adds learning stability, this is offset by the reduced learning speed, leading to frequent loss-of-control due to the inherent dynamic instability of rotorcraft. After a 120 second online training phase, the resulting controller was shown to be able to perform two different, aggressive manoeuvres when provided with a proper reference signal.  It can be concluded that the proposed framework is a first step towards a helicopter flight control system based on online reinforcement learning.

To further advance this field, further research is recommended. Firstly, the control system should be expanded to control all four axes instead of only the longitudial motions. Although it was not found in this research, it is also speculated that a modification of the current setup where multiple control inputs are controlled by one agent could improve the performance of the control system.  The desired performance characteristics of the ADS-33 acceleration-deceleration manoeuvre could not be achieved because of unwanted collective-pedal coupling effects which are suggested to be due to modeling deficiencies. Therefore, the use of a higher-fidelity helicopter model with more degrees of freedom is suggested. Finally, the assumptions of clean measurements and no turbulence done in this research are not realistic. Future research should work on quantifying the effects of these two disturbances. 